\documentclass[]{article}
\usepackage{lmodern}
\usepackage{amssymb,amsmath}
\usepackage{ifxetex,ifluatex}
\usepackage{fixltx2e} % provides \textsubscript
\ifnum 0\ifxetex 1\fi\ifluatex 1\fi=0 % if pdftex
  \usepackage[T1]{fontenc}
  \usepackage[utf8]{inputenc}
\else % if luatex or xelatex
  \ifxetex
    \usepackage{mathspec}
  \else
    \usepackage{fontspec}
  \fi
  \defaultfontfeatures{Ligatures=TeX,Scale=MatchLowercase}
\fi
% use upquote if available, for straight quotes in verbatim environments
\IfFileExists{upquote.sty}{\usepackage{upquote}}{}
% use microtype if available
\IfFileExists{microtype.sty}{%
\usepackage{microtype}
\UseMicrotypeSet[protrusion]{basicmath} % disable protrusion for tt fonts
}{}
\usepackage[margin=1in]{geometry}
\usepackage{hyperref}
\hypersetup{unicode=true,
            pdftitle={Lecture 22},
            pdfauthor={Wesley Burr},
            pdfborder={0 0 0},
            breaklinks=true}
\urlstyle{same}  % don't use monospace font for urls
\usepackage{color}
\usepackage{fancyvrb}
\newcommand{\VerbBar}{|}
\newcommand{\VERB}{\Verb[commandchars=\\\{\}]}
\DefineVerbatimEnvironment{Highlighting}{Verbatim}{commandchars=\\\{\}}
% Add ',fontsize=\small' for more characters per line
\usepackage{framed}
\definecolor{shadecolor}{RGB}{248,248,248}
\newenvironment{Shaded}{\begin{snugshade}}{\end{snugshade}}
\newcommand{\AlertTok}[1]{\textcolor[rgb]{0.94,0.16,0.16}{#1}}
\newcommand{\AnnotationTok}[1]{\textcolor[rgb]{0.56,0.35,0.01}{\textbf{\textit{#1}}}}
\newcommand{\AttributeTok}[1]{\textcolor[rgb]{0.77,0.63,0.00}{#1}}
\newcommand{\BaseNTok}[1]{\textcolor[rgb]{0.00,0.00,0.81}{#1}}
\newcommand{\BuiltInTok}[1]{#1}
\newcommand{\CharTok}[1]{\textcolor[rgb]{0.31,0.60,0.02}{#1}}
\newcommand{\CommentTok}[1]{\textcolor[rgb]{0.56,0.35,0.01}{\textit{#1}}}
\newcommand{\CommentVarTok}[1]{\textcolor[rgb]{0.56,0.35,0.01}{\textbf{\textit{#1}}}}
\newcommand{\ConstantTok}[1]{\textcolor[rgb]{0.00,0.00,0.00}{#1}}
\newcommand{\ControlFlowTok}[1]{\textcolor[rgb]{0.13,0.29,0.53}{\textbf{#1}}}
\newcommand{\DataTypeTok}[1]{\textcolor[rgb]{0.13,0.29,0.53}{#1}}
\newcommand{\DecValTok}[1]{\textcolor[rgb]{0.00,0.00,0.81}{#1}}
\newcommand{\DocumentationTok}[1]{\textcolor[rgb]{0.56,0.35,0.01}{\textbf{\textit{#1}}}}
\newcommand{\ErrorTok}[1]{\textcolor[rgb]{0.64,0.00,0.00}{\textbf{#1}}}
\newcommand{\ExtensionTok}[1]{#1}
\newcommand{\FloatTok}[1]{\textcolor[rgb]{0.00,0.00,0.81}{#1}}
\newcommand{\FunctionTok}[1]{\textcolor[rgb]{0.00,0.00,0.00}{#1}}
\newcommand{\ImportTok}[1]{#1}
\newcommand{\InformationTok}[1]{\textcolor[rgb]{0.56,0.35,0.01}{\textbf{\textit{#1}}}}
\newcommand{\KeywordTok}[1]{\textcolor[rgb]{0.13,0.29,0.53}{\textbf{#1}}}
\newcommand{\NormalTok}[1]{#1}
\newcommand{\OperatorTok}[1]{\textcolor[rgb]{0.81,0.36,0.00}{\textbf{#1}}}
\newcommand{\OtherTok}[1]{\textcolor[rgb]{0.56,0.35,0.01}{#1}}
\newcommand{\PreprocessorTok}[1]{\textcolor[rgb]{0.56,0.35,0.01}{\textit{#1}}}
\newcommand{\RegionMarkerTok}[1]{#1}
\newcommand{\SpecialCharTok}[1]{\textcolor[rgb]{0.00,0.00,0.00}{#1}}
\newcommand{\SpecialStringTok}[1]{\textcolor[rgb]{0.31,0.60,0.02}{#1}}
\newcommand{\StringTok}[1]{\textcolor[rgb]{0.31,0.60,0.02}{#1}}
\newcommand{\VariableTok}[1]{\textcolor[rgb]{0.00,0.00,0.00}{#1}}
\newcommand{\VerbatimStringTok}[1]{\textcolor[rgb]{0.31,0.60,0.02}{#1}}
\newcommand{\WarningTok}[1]{\textcolor[rgb]{0.56,0.35,0.01}{\textbf{\textit{#1}}}}
\usepackage{graphicx,grffile}
\makeatletter
\def\maxwidth{\ifdim\Gin@nat@width>\linewidth\linewidth\else\Gin@nat@width\fi}
\def\maxheight{\ifdim\Gin@nat@height>\textheight\textheight\else\Gin@nat@height\fi}
\makeatother
% Scale images if necessary, so that they will not overflow the page
% margins by default, and it is still possible to overwrite the defaults
% using explicit options in \includegraphics[width, height, ...]{}
\setkeys{Gin}{width=\maxwidth,height=\maxheight,keepaspectratio}
\IfFileExists{parskip.sty}{%
\usepackage{parskip}
}{% else
\setlength{\parindent}{0pt}
\setlength{\parskip}{6pt plus 2pt minus 1pt}
}
\setlength{\emergencystretch}{3em}  % prevent overfull lines
\providecommand{\tightlist}{%
  \setlength{\itemsep}{0pt}\setlength{\parskip}{0pt}}
\setcounter{secnumdepth}{0}
% Redefines (sub)paragraphs to behave more like sections
\ifx\paragraph\undefined\else
\let\oldparagraph\paragraph
\renewcommand{\paragraph}[1]{\oldparagraph{#1}\mbox{}}
\fi
\ifx\subparagraph\undefined\else
\let\oldsubparagraph\subparagraph
\renewcommand{\subparagraph}[1]{\oldsubparagraph{#1}\mbox{}}
\fi

%%% Use protect on footnotes to avoid problems with footnotes in titles
\let\rmarkdownfootnote\footnote%
\def\footnote{\protect\rmarkdownfootnote}

%%% Change title format to be more compact
\usepackage{titling}

% Create subtitle command for use in maketitle
\newcommand{\subtitle}[1]{
  \posttitle{
    \begin{center}\large#1\end{center}
    }
}

\setlength{\droptitle}{-2em}

  \title{Lecture 22}
    \pretitle{\vspace{\droptitle}\centering\huge}
  \posttitle{\par}
    \author{Wesley Burr}
    \preauthor{\centering\large\emph}
  \postauthor{\par}
      \predate{\centering\large\emph}
  \postdate{\par}
    \date{27/11/2019}


\begin{document}
\maketitle

\hypertarget{warmup-example-cyber-security}{%
\subsection{Warmup Example: Cyber
Security}\label{warmup-example-cyber-security}}

Based on information from the National Cyber Security Alliance, 93\% of
computer owners believe they have antivirus programs installed on their
computers.

In a random sample of 400 scanned computers, it is found that 380 of
them (or 95\%) actually have antivirus software programs.

Use the sample data from the scanned computers to test the claim that
93\% of computers have antivirus software.

\hypertarget{requirements}{%
\subsection{Requirements}\label{requirements}}

\begin{enumerate}
\def\labelenumi{\arabic{enumi}.}
\tightlist
\item
  The 400 computers were randomly selected (check!)
\item
  There is a fixed number of independent trials, two possible outcomes
  (check!)
\item
  Is \(np \geq 10\)? Is \(n(1-p) \geq 10\)? \[
  \begin{aligned*}
   np &= (400)(0.93) = 372 \\
  n(1-p) &= (400)(1-0.93) = 28
  \end{aligned*}
  \] Check!
\end{enumerate}

\hypertarget{hypotheses}{%
\subsection{Hypotheses}\label{hypotheses}}

Write the hypotheses:

\[
\mathbf{H_0}: p = 0.93 \qquad \text{versus} \qquad \mathbf{H_A}: p \neq 0.93
\]

\hypertarget{significance-level-test-statistic}{%
\subsection{Significance Level, Test
Statistic}\label{significance-level-test-statistic}}

Since we didn't have a specified level, choose \(\alpha = 0.05\). We are
testing a claim about a \textbf{population proportion}, so we will use a
normal approximation:

\[
z_\text{test} = \frac{\hat{p} - p_0}{\sqrt{\frac{p_0(1-p_0)}{n}}} = \frac{ \frac{380}{400} - 0.93}{\sqrt{\frac{0.93(0.07)}{400}}}
\]

\begin{Shaded}
\begin{Highlighting}[]
\NormalTok{z_test <-}\StringTok{ }\NormalTok{( }\DecValTok{380}\OperatorTok{/}\DecValTok{400} \OperatorTok{-}\StringTok{ }\FloatTok{0.93}\NormalTok{ ) }\OperatorTok{/}\StringTok{ }\KeywordTok{sqrt}\NormalTok{( }\FloatTok{0.93} \OperatorTok{*}\StringTok{ }\FloatTok{0.07} \OperatorTok{/}\StringTok{ }\DecValTok{400}\NormalTok{ )}
\NormalTok{z_test}
\end{Highlighting}
\end{Shaded}

\begin{verbatim}
## [1] 1.567724
\end{verbatim}

\hypertarget{the-p-value}{%
\subsection{The p-value}\label{the-p-value}}

\begin{Shaded}
\begin{Highlighting}[]
\KeywordTok{pnorm}\NormalTok{(z_test, }\DataTypeTok{lower.tail =} \OtherTok{FALSE}\NormalTok{) }\OperatorTok{*}\StringTok{ }\DecValTok{2}
\end{Highlighting}
\end{Shaded}

\begin{verbatim}
## [1] 0.1169457
\end{verbatim}

\includegraphics{Lecture22_files/figure-latex/unnamed-chunk-3-1.pdf}

\hypertarget{conclusion}{%
\subsection{Conclusion}\label{conclusion}}

Thus, since \(p > \alpha\), we do not have evidence at the 95\% level to
conclude that the population proportion of computers having antivirus
software is not 93\%. In other words, there is not sufficient evidence
to warrant rejection of this claim.

\hypertarget{solving-problems}{%
\section{Solving Problems}\label{solving-problems}}

\hypertarget{three-kinds-of-problems}{%
\subsection{Three Kinds of Problems}\label{three-kinds-of-problems}}

There are three main kinds of problems we've learned about, each with
one or two sub-types.

\begin{itemize}
\tightlist
\item
  Normal distribution, question about means: confidence intervals (Z)
  and hypothesis tests (Z)

  \begin{itemize}
  \tightlist
  \item
    Requires \textbf{either} known sigma and normality \textbf{or}
    normality and 30+ samples
  \end{itemize}
\item
  t distribution, question about means: confidence intervals (t) and
  hypothesis tests (t)

  \begin{itemize}
  \tightlist
  \item
    The rest of the cases: less than 30 samples \textbf{and} sigma not
    known
  \item
    Technically not all the cases: you'll discuss this more if you take
    1052H
  \end{itemize}
\item
  Normal distribution, question about proportions: confidence intervals
  (Z) and hypothesis tests (Z)
\end{itemize}

\hypertarget{flow-chart}{%
\subsection{Flow Chart}\label{flow-chart}}

\hypertarget{examples}{%
\section{Examples}\label{examples}}

\hypertarget{problem-1-discarded-plastics}{%
\subsection{Problem 1: Discarded
Plastics}\label{problem-1-discarded-plastics}}

A sample of 62 households had their recycling audited. The weights of
their weekly recycled plastic had sample mean 1.911 pounds, with sample
standard deviation 1.065 pounds. At the 95\% level, test the claim that
the mean weight of recycled plastic from the population of households is
greater than 1.80 pounds.

Also find a 95\% confidence interval for the mean weight of recycled
plastic per week.

\hypertarget{section}{%
\subsection{\texorpdfstring{\(\;\)}{\textbackslash{};}}\label{section}}

\hypertarget{section-1}{%
\subsection{\texorpdfstring{\(\;\)}{\textbackslash{};}}\label{section-1}}

\hypertarget{problem-2-the-ysort-trial}{%
\subsection{Problem 2: The YSORT
Trial}\label{problem-2-the-ysort-trial}}

The Genetics and IVF Institute conducted a clinical trial of the YSORT
method designed to increase the probability of conceiving a boy. In this
trial, 291 babies were born to parents using the YSORT method, and 239
of them were boys. Use a 99\% significance level to test the claim that
the YSORT method is effective at increasing the likelihood that a baby
will be a boy.

Also find a 95\% confidence interval for the true underlying proportion
of babies born using the YSORT method who will be boys.

\hypertarget{section-2}{%
\subsection{\texorpdfstring{\(\;\)}{\textbackslash{};}}\label{section-2}}

\hypertarget{section-3}{%
\subsection{\texorpdfstring{\(\;\)}{\textbackslash{};}}\label{section-3}}

\hypertarget{problem-3-highway-speeds}{%
\subsection{Problem 3: Highway Speeds}\label{problem-3-highway-speeds}}

Southbound traffic on the I-280 highway near Cupertino, California had
its speed monitored at 3:30pm on a Wednesday. The sample of 12 cars had
mean 97.6 km/hr with standard deviation 6.56 km/hr. Test the highway
patrol's claim that the average speed on this highway at this time of
day is lower than the speed limit of 105 km/hr.

Also compute a 99\% confidence interval for the mean speed.

\hypertarget{section-4}{%
\subsection{\texorpdfstring{\(\;\)}{\textbackslash{};}}\label{section-4}}

\hypertarget{section-5}{%
\subsection{\texorpdfstring{\(\;\)}{\textbackslash{};}}\label{section-5}}

\hypertarget{problem-4-weights-of-pennies}{%
\subsection{Problem 4: Weights of
Pennies}\label{problem-4-weights-of-pennies}}

Before 1983, US pennies were made with 97\% copper and 3\% zinc. After
1983, they were converted to 3\% copper and 97\% zinc to make them
cheaper to manufacture. A simple random sample of 35 post-1983 pennies
had an average weight of 2.49910g, with standard deviation 0.01648g. The
US Mint specifies that post-1983 pennies should be manufactured with
mean weight 2.500g. At a 95\% level, do you believe that pennies are
actually being manufactured with mean weight of 2.500g?

Compute a 99\% confidence interval for the mean weight of post-1983
pennies.

\hypertarget{section-6}{%
\subsection{\texorpdfstring{\(\;\)}{\textbackslash{};}}\label{section-6}}

\hypertarget{section-7}{%
\subsection{\texorpdfstring{\(\;\)}{\textbackslash{};}}\label{section-7}}

\hypertarget{problem-5-cell-phones-and-cancer}{%
\subsection{Problem 5: Cell Phones and
Cancer}\label{problem-5-cell-phones-and-cancer}}

In a study of 420,095 Danish cell phone users, 125 subjects developed
cancer of the brain or nervous system (Journal of the National Cancer
Institute). Test the claim of the belief that such cancers are affected
by cell phone use. That is, test the claim that cell phone users develop
cancer of the brain or nervous system at a rate that is different from
the rate of 0.0340\% for people who do not use cell phones. Use a 99.5\%
significance level.

\hypertarget{section-8}{%
\subsection{\texorpdfstring{\(\;\)}{\textbackslash{};}}\label{section-8}}

\hypertarget{section-9}{%
\subsection{\texorpdfstring{\(\;\)}{\textbackslash{};}}\label{section-9}}

\hypertarget{problem-6-insomnia-and-zopiclone}{%
\subsection{Problem 6: Insomnia and
Zopiclone}\label{problem-6-insomnia-and-zopiclone}}

A clinical trial was conducted to test the effectiveness of the drug
Zopiclone for treating insomnia in older subjects. Before treatment with
Zopiclone, 16 subjects had a mean wake time of 102.8 minutes. After
treatment with Zopiclone, the 16 subjects had a mean wake time of 98.9
minutes, and a standard deviation of 42.3 minutes (JAMA). Assume that
the 16 sample values appear to be from a normally distributed
population, and test the claim that after treatment with Zopiclone,
subjects have a reduced mean wake time.

Also find a 95\% confidence interval for the mean wake time.

\hypertarget{section-10}{%
\subsection{\texorpdfstring{\(\;\)}{\textbackslash{};}}\label{section-10}}

\hypertarget{section-11}{%
\subsection{\texorpdfstring{\(\;\)}{\textbackslash{};}}\label{section-11}}


\end{document}
